
\documentstyle[12pt]{article}

\begin{document}
\section{The Graphics Program PLAUT}

\index{PLAUT} is a program that can be used to extract graphical
information from the AUTO output files {\tt fort.7} and {\tt fort.8}.
These files are referred to below as Unit 7 and Unit 8, respectively.
When saved, these files are called {\tt p.xxx} and {\tt q.xxx}, respectively,
 where {\tt xxx} stands for a user selected name.

\subsection{The Principal PLAUT Commands}

The principal PLAUT commands are 

\begin{description}
\item[\bf BD0 ---] This command is useful for an initial overview of the bifurcation
                   diagram as stored in Unit~7.
                   If you have not previously selected one of the default options 
                   (D0, D1, D2, D3, or D4) described below, then you will be asked
                   whether you want solution labels, grid lines, titles, or labeled axes.

\item[\bf BD ---]  This command is the same as the BD0 command, except that you will be
                   asked to enter the minimum and the maximum of the horizontal and 
                   vertical axes.
                   This is useful for blowing up portions of a previously displayed
                   bifurcation diagram.

\item[\bf AX ---]  With the AX command you can select any pair of columns of real
                   numbers from Unit 7 as horizontal and vertical axis in the
                   bifurcation diagram. (The default is columns 1 and 2).
                   To determine what these columns represent, one can look at the
                   screen ouput of the corresponding AUTO run, or one can inspect the
                   column headings in the Unit~7 file.
                                 
\item[\bf 2D ---]  Upon entering the 2D command, the labels of all solutions stored 
                   in Unit~8 will be listed and you can select one or more of these 
                   for display. The number of solution components is also listed
                   and you will be prompted to select two of these as horizontal and
                   vertical axis in the display.
                   Note that the first component is typically the independent 
                   time or space variable scaled to the interval [0,1].

\item[\bf SAV ---] To save the displayed plot in a file. You will be asked to enter
                   a file name. Each plot must be stored in a separate new file.
                   The plot is stored in compact PLOT10 format, which can be converted to 
                   Postscript format with the AUTO {\tt @ps} and {\tt @pr}
                   commands (See Section ???.)

\item[\bf CL ---]  To clear the graphics window.

\item[\bf LAB ---]  To list the labels of all solutions stored in Unit 8.
                    Note that PLAUT requires all labels to be distinct.
                    In case of multiple labels you can use the AUTO
                    command {\tt @lb} to relabel solutions in
                    Unit~7 and Unit~8.

\item[\bf END ---]  To end execution of PLAUT.
\end{description}


\subsection{Default Options}

After entering the commands BD0, BD, or 2D, you will be asked whether you 
want solution labels, grid lines, titles, or axes labels.
For quick plotting it is convenient to bypass these selections.
This can be done by the default commands D0, D1, D2, D3, or D4 below.
These can be entered as a single command before the BD0 or BD command,
or they can be entered as prefixes in the BD0 or BD command. 
Thus, for example, one can enter the command D1BD0.  

\begin{description}
\item[\bf D0 ---]  Use solid curves, showing solution labels and symbols.    
\item[\bf D1 ---]  Use solid curves, except use dashed curves for unstable
                   solutions and for solutions of unknown stability.
                   Show solution labels and symbols.
\item[\bf D2 ---]  As D1, but with grid lines.   
\item[\bf D3 ---]  As D1, except for periodic solutions use 
                   solid circles if stable,
                   and open circles if unstable or if the stability
                   is unknown.
\item[\bf D4 ---]  Use solid curves, without labels and symbols.  
\end{description}

If no default option (D0, D1, D2, D3, or D4) has been selected 
or if you want to override a default feature,
then the the following commands can be used.
These can be entered as individual commands or as prefixes.
For example, one can enter the command SYDPBD0.

\begin{description}     
\item[\bf SY ---]    Use symbols for special solution points, for example,
                     open square = bifurcation,
                     solid square = Hopf bifurcation.
\item[\bf DP ---]    ``Differential Plots'', i.e., show stability of the 
                     solutions. Solid curves represent stable solutions.
                     Dashed curves are used for unstable
                     solutions and for solutions of unknown stability.
                     For periodic solutions use solid and open circles
                     to indicate stability/instability (or unknown
                     stability).
\item[\bf ST ---]    Set up titles and axes labels. 
\item[\bf NU ---]    Normal usage (reset special options). 
\end{description}


\subsection{Other PLAUT Commands}

The full PLAUT program has several other capabilities.
The most useful of the additional commands are

\begin{description}     
\item[\bf SCR ---]  To change the diagram size.
\item[\bf RSS ---]  To change the size of special point symbols.
\end{description}


\subsection{Printing PLAUT Files}
 
\begin{description}

\item[{\tt @ps} ---]
To convert a PLAUT file {\tt fig.x} to {\sc Post}{\sc Script} format
as {\tt fig.x.ps}.

\item[{\tt @pr} ---]
To convert a PLAUT file {\tt fig.x} to {\sc Post}{\sc Script}, and to
send the resulting file {\tt fig.x.ps} to the printer.

\end{description}

\end{document}
